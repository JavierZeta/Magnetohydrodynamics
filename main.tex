\documentclass[12pt,a4paper]{article}


\usepackage[spanish]{babel}
\usepackage[utf8]{inputenc}
\usepackage[T1]{fontenc}



\usepackage{amsmath, amssymb, bm}
\usepackage{physics}
\usepackage{geometry}
\usepackage{setspace}
\usepackage{graphicx}
\usepackage{caption}
\usepackage{float}
\usepackage{hyperref}
\usepackage{xcolor}

\geometry{
  left=3cm,
  right=3cm,
  top=3cm,
  bottom=3cm
}

\onehalfspacing


\newcommand{\vect}[1]{\boldsymbol{#1}}

\title{\textbf{The Cosmic Engine}\\
\large Fundamentos Magnetohidrodinámicos de la Acreción y la Eyección Relativista}
\author{ }
\date{ }

\begin{document}

\maketitle
\thispagestyle{empty}
\newpage

\tableofcontents
\newpage
% =========================================================
\section{Introducción a la acreción: El problema del momento angular y por qué la viscosidad clásica falla}
% =========================================================

Los discos de acreción son objetos astrofísicos formados principalmente por gas que cuenta con un momento angular respecto a un objeto central de masa $M$ al que orbitan. Gracias a este momento angular, no todo el gas acaba cayendo al objeto y se queda orbitando, aplanándolo y dándole la forma de disco. Un ejemplo muy conocido es el disco de acreción que se forma en las estrellas T-Tauri en las primeras etapas de formación de una estrella. Para facilitar la visualización de este objeto se incluye la siguiente imagen. 
\begin{figure}[h]
    \centering
    \includegraphics[width=0.4\linewidth]{Fotos/DA.jpg}
    \caption{Disco de acreción de una estrella T-Tauri, la cual se encuentra en el centro}
\end{figure} \\
En las estrellas T-Tauri, el disco de acreción proporciona el material necesario para que la estrella siga ganando masa mediante el mecanismo de acreción. Esto plantea una problemática, ya que si el gas tiene suficiente momento angular como para orbitar la estrella, ¿cómo es posible que parte de ese material caiga a la estrella? Para ello es necesario que el gas pierda momento angular y así desplazarse hacia radios menores. Esto implica la existencia de mecanismos de transporte del momento angular dentro del disco.

% ---------------------------------------------------------
\subsection{El problema del momento angular}
% ---------------------------------------------------------

Vamos a considerar una partícula de gas del disco de acreción con masa $m$, velocidad $\vect{v}$ y una distancia al objeto que orbita $\vect{r}$.  Podemos calcular su momento angular del siguiente modo:
\begin{equation}
    \vect{L}=\vect{r}\times\vect{p}=\vect{r}\times(m\vect{v})
\end{equation}
La derivada temporal del momento angular es igual al torque, el cual podemos definir como:
\begin{equation}
    \vect{\tau}=\vect{r}\times\vect{F_{g}}
\end{equation}
Sin embargo, la fuerza gravitatoria es una fuerza central, por lo que podemos reescribirla como $\vect{F_{g}}=F(r)\vect{\hat{r}}$. Esto quiere decir que:
\begin{equation}
    \frac{d\vect{L}}{dt}=\vect{r}\times(F(r)\vect{\hat{r}})\xrightarrow{\vect{r}||\vect{\hat{r}}}\frac{d\vect{L}}{dt}=0
\end{equation}
Esto implica que $\vect{L}$ es un vector constante tanto en magnitud como en dirección y, por tanto, esta partícula de gas que estamos estudiando jamás podría ser acretada por el objeto central. Con lo cual, es necesario que exista un torque adicional externo que provoque que las partículas más cercanas al objeto central pierdan el suficiente momento angular para que sean acretadas y las partículas más exteriores lo ganen.

% ---------------------------------------------------------
\subsection{Viscosidad como torque}
% ---------------------------------------------------------

Como ya se concluyó en la última subsección, es necesario que exista una fuerza que permita que el fluido del disco pierda momento angular. Considerando un fluido clásico, esta fuerza viene de la llamada viscosidad molecular $\nu_{m}$, el cual surge de las colisiones e intercambio de la cantidad de movimiento del gas.
\\\
El tiempo característico en el que la viscosidad molecular actúa se define como:
\begin{equation}
    t_{\nu}=\frac{R^{2}}{\nu_{m}}
\end{equation}
Donde $R$ es el radio del disco de acreción. Para ver si esta viscosidad molecular es capaz de explicar la acreción, vamos a calcular $t_{\nu}$ y compararlo con el tiempo que dura esta fase de acreción usando de nuevo el ejemplo de las estrellas T-Tauri. 
\\\
Primero de todo, vamos a calcular el valor de la viscosidad molecular para el gas del disco. Esta viscosidad se puede calcular del siguiente modo:
\begin{equation}
    \nu_{m}=lc_{s}
\end{equation}
Donde $l$ es el recorrido libre medio de una partícula y $c_{s}$ es la velocidad del sonido. El recorrido libre medio se puede obtener como:
\begin{equation}
    l\sim\frac{1}{\sigma n}
\end{equation}
Siendo $n$ el número de partículas por unidad de volumen y $\sigma$ la sección eficaz. Teniendo en cuenta que, en su mayoría, los discos de acreción están formados por hidrógeno molecular ($H_{2}$) tenemos los siguiente valores: $n=10^{13}\,cm^{-3}$ y $\sigma=10^{-15}\,cm^{2}$. Con lo cual:
\begin{equation}
    l\sim100\,cm=1\,m
\end{equation}
Con esto y teniendo en cuenta que la velocidad del sonido es de $c_{s}=10^{3}\,m/s$ a unos $T=300\,K$, obtenemos el siguiente valor para la viscosidad molecular:
\begin{equation}
    \nu_{m}=lc_{s}=10^{3}m^{2}/s
\end{equation}
Los radios típicos de un disco de acreción en estrellas T-Tauri son de $R=1\,ua=1,49\cdot10^{11}\,m$. Entonces:
\begin{equation}
    t_{\nu}=\frac{R^{2}}{\nu_{m}}\sim2,22\cdot10^{19}\,s=7,04\cdot10^{11}\,yr
\end{equation}
Este es un tiempo enorme y, de hecho, es mayor a la edad del Universo, por lo que la viscosidad molecular clásica no puede explicar la formación de discos de acreción. Además, se sabe también que, en estos discos, el número de Reynolds es bastante elvado ($Re>10^{10}$), por lo que las fuerzas moleculares viscosas se pueden despreciar frente a las fuerzas inerciales.
\\\
Para explicar la acreción se han propuesto diferentes modelos basados que postulan una viscosidad efectiva basadas en otros fenómenos físicos como las turbulencias o las inestabilidades magnéticas. Hoy en día, el modelo más utilizado es el conocido como modelo $\alpha$ de Shakura-Sunyaev, donde la viscosidad se parametriza como $\nu=\alpha c_{s}H$. Siendo $H$ el espesor del disco y $\alpha$ es un parámetro menor a 1.
% =========================================================
\section{Fundamentos de la Magnetohidrodinámica Ideal}
% =========================================================

La Magnetohidrodinámica (MHD) proporciona el marco teórico fundamental para describir la interacción entre campos magnéticos y plasmas conductores en sistemas astrofísicos. En este capítulo se introducen los principios físicos y matemáticos que permiten entender cómo un campo magnético inicialmente débil puede ejercer un control dinámico sobre discos de acreción, preparando el terreno para el estudio de la inestabilidad magnetorrotacional en el Capítulo~3.

% ---------------------------------------------------------
\subsection{La unión entre Mecánica de Fluidos y Electromagnetismo}
% ---------------------------------------------------------

La MHD puede entenderse, siguiendo a Davidson, como una teoría efectiva que emerge de la combinación de la Mecánica de Fluidos clásica con el Electromagnetismo de Maxwell.
\\\\
Desde la parte hidrodinámica, el plasma se modela como un fluido continuo caracterizado por campos macroscópicos: densidad de masa $\rho(\vect{x},t)$, velocidad del fluido $\vect{u}(\vect{x},t)$ y presión escalar $p(\vect{x},t)$.Esta descripción hidrodinámica es válida siempre que exista una clara separación entre las escalas microscópicas del plasma y las escalas macroscópicas del sistema, de modo que las cantidades físicas relevantes puedan definirse como promedios locales continuos. En particular, se requiere que la longitud libre media de las partículas y las escalas cinéticas asociadas sean mucho menores que las escalas características de variación espacial y temporal del flujo, lo que permite ignorar la dinámica individual de las partículas y trabajar con campos macroscópicos bien definidos.
\\\\
Asimismo, se asume que el plasma es cuasi-neutro a escalas macroscópicas, lo que implica que las separaciones locales de carga quedan confinadas a escalas del orden de la longitud de Debye y no influyen en la dinámica global. 
\\\\
Además necesitamos que el plasma sea colisionalmente eficiente o, de forma más general, que exista algún mecanismo físico que asegure un equilibrio termodinámico local. Bajo esta hipótesis, la distribución de velocidades puede aproximarse por una distribución cercana a la Maxwelliana, lo que permite definir una presión escalar bien definida y cerrar el sistema de ecuaciones con un número finito de momentos.
\\\\
Se supone también que la presión es aproximadamente isotrópica, lo que implica que las anisotropías asociadas al campo magnético o a trayectorias preferentes han sido suprimidas por colisiones, turbulencia u otros procesos de dispersión. Esta condición permite representar la presión mediante un escalar y evita la introducción de tensores de presión anisótropos.
\\\\
Finalmente, se considera que las velocidades características del flujo son no relativistas, de modo que los efectos relativistas pueden despreciarse en la formulación de las ecuaciones de conservación y en el acoplamiento con el Electromagnetismo.
\\\\
Por el lado electromagnético, el punto de partida son las ecuaciones de Maxwell:
\begin{align}
\nabla \cdot \vect{E} &= \frac{\rho_e}{\varepsilon_0}  , \\
\nabla \cdot \vect{B} &= 0, \\
\nabla \times \vect{E} &= -\frac{\partial \vect{B}}{\partial t} \label{eq:maxwell_faraday}, \\
\nabla \times \vect{B} &= \mu_0 \vect{J} + \mu_0 \varepsilon_0 \frac{\partial \vect{E}}{\partial t}. \label{eq:maxwell_ampere}
\end{align}

Por las condiciones de cuasi-neutralidad y régimen no relativista podemos anular los términos 
\begin{equation}
\rho_e \approx 0  \textit{  y  } \mu_0 \varepsilon_0 \partial \vect{E}/\partial t\approx 0
\end{equation}

 Bajo estas hipótesis, la ley de Ampère se simplifica y el Electromagnetismo queda dominado por la interacción entre corrientes inducidas y campos magnéticos.

% ---------------------------------------------------------
\subsection{La ecuación de inducción magnética}
% ---------------------------------------------------------



De acuerdo con las hipótesis del modelo MHD, el movimiento del fluido es no
relativista. Bajo esta suposición, es posible describir localmente el campo
electromagnético en el sistema de referencia que se mueve con el fluido, empleando
una transformación galileana con velocidad $\vect{u}$.

Denotando con un primo las magnitudes medidas en el marco del fluido, las
transformaciones galileanas de los campos electromagnéticos toman la forma:
\begin{align}
\vect{B}' &= \vect{B}, \\
\vect{J}' &= \vect{J}, \\
\vect{E}' &= \vect{E} + \frac{1}{c}\, \vect{u} \times \vect{B}.
\end{align}

Dado que el plasma se comporta como un fluido conductor, la corriente eléctrica es
impulsada por el campo eléctrico medido en el marco del fluido. Para establecer una
relación constitutiva entre la corriente y el campo eléctrico, se introduce la
siguiente hipótesis.

\medskip
\noindent
\textbf{Hipótesis (Ley de Ohm en el marco del fluido).}
Se adopta la forma más simple de la relación corriente--campo, consistente con el
modelo magnetohidrodinámico de un solo fluido, asumiendo una conductividad escalar
constante:
\begin{equation}
\vect{J}' = \sigma \vect{E}'.
\label{eq:ohm_fluid}
\end{equation}

Sustituyendo la transformación galileana del campo eléctrico en la expresión
(\ref{eq:ohm_fluid}), se obtiene la ley de Ohm expresada en el sistema de referencia
del laboratorio:
\begin{equation}
\vect{E} = \frac{\vect{J}}{\sigma} - \frac{1}{c}\, \vect{u} \times \vect{B}.
\label{eq:ohm_lab}
\end{equation}


Sustituyendo las ecuaciones (\ref{eq:ohm_lab}) y (\ref{eq:maxwell_ampere}) en la
ecuación de Faraday~(\ref{eq:maxwell_faraday}), se obtiene:
\begin{equation}
\frac{\partial \vect{B}}{\partial t}
=
\nabla \times (\vect{u} \times \vect{B})
- \frac{1}{\mu_0 \sigma}\, \nabla \times (\nabla \times \vect{B}).
\end{equation}

\medskip
\noindent
\textbf{ Usando la Identidad vectorial:}
\begin{equation}
\nabla \times (\nabla \times \vect{B})
=
\nabla (\nabla \cdot \vect{B}) - \nabla^2 \vect{B}.
\end{equation}

Dado que $\nabla \cdot \vect{B} = 0$, condición impuesta por la ecuación de Maxwell
correspondiente, el primer término se anula. Definiendo la difusividad magnética
\begin{equation}
\eta \equiv \frac{1}{\mu_0 \sigma},
\end{equation}
se obtiene finalmente la ecuación de inducción magnética:
\begin{equation}
\boxed{
\frac{\partial \vect{B}}{\partial t}
=
\nabla \times (\vect{u} \times \vect{B})
+ \eta \nabla^2 \vect{B}
}
\label{eq:induction}
\end{equation}

% ---------------------------------------------------------
\subsection{El teorema de Alfv\'en y el concepto de flujo congelado}
% ---------------------------------------------------------

Una de las consecuencias más profundas de la ecuación de inducción magnética es el
denominado \textit{teorema de Alfv\'en}, también conocido como el principio de
\textit{flujo congelado}. Este teorema establece que, en el régimen de MHD ideal, las
líneas de campo magnético se mueven conjuntamente con el fluido conductor, como si
estuvieran ``congeladas'' en él.

El punto de partida es la ecuación de inducción magnética,
ecuación~(\ref{eq:induction}). En el límite de conductividad infinita,
$\sigma \rightarrow \infty$, la difusividad magnética $\eta$ tiende a cero y la
ecuación se reduce a:
\begin{equation}
\frac{\partial \vect{B}}{\partial t}
=
\nabla \times (\vect{u} \times \vect{B}).
\label{eq:induction_ideal}
\end{equation}

Consideremos ahora una superficie material $S(t)$ que se mueve con el fluido,
delimitada por un contorno cerrado $C(t)$. El flujo magnético a través de dicha
superficie se define como
\begin{equation}
\Phi_B(t) = \int_{S(t)} \vect{B} \cdot \mathrm{d}\vect{S}.
\end{equation}

Utilizando el teorema de transporte
de Reynolds para una superficie material, se obtiene:
\begin{equation}
\frac{\mathrm{d}\Phi_B}{\mathrm{d}t}
=
\int_{S(t)} \frac{\partial \vect{B}}{\partial t} \cdot \mathrm{d}\vect{S}
+
\int_{S(t)} \nabla \cdot (\vect{u}\vect{B}) \cdot \mathrm{d}\vect{S}
-
\int_{S(t)} (\vect{B}\cdot\nabla)\vect{u} \cdot \mathrm{d}\vect{S}.
\end{equation}

Sustituyendo la ecuación de inducción ideal (\ref{eq:induction_ideal}) y empleando
identidades vectoriales triviales, la expresión anterior puede reorganizarse como:
\begin{equation}
\frac{\mathrm{d}\Phi_B}{\mathrm{d}t}
=
\int_{S(t)} \left[
\nabla \times (\vect{u} \times \vect{B})
- \nabla \times (\vect{u} \times \vect{B})
\right] \cdot \mathrm{d}\vect{S}.
\end{equation}

Por lo tanto, se obtiene inmediatamente:
\begin{equation}
\frac{\mathrm{d}\Phi_B}{\mathrm{d}t} = 0.
\end{equation}

Este resultado establece que, en MHD ideal, el flujo magnético a través de cualquier
superficie material permanece constante en el tiempo. En consecuencia, las líneas de
campo magnético son transportadas por el fluido y conservan su conectividad
topológica.

Desde un punto de vista físico, las líneas de campo se comportan como entidades
materiales: pueden estirarse y deformarse por el movimiento del plasma, pero no
romperse ni reconectarse mientras la resistividad sea despreciable. Este principio
constituye la base conceptual del transporte de momento angular en discos de
acreción y de la inestabilidad magnetorrotacional.
\\\\
\textbf{Número de Reynolds magnético.}

El grado de validez del régimen de flujo congelado puede cuantificarse mediante el
\textit{número de Reynolds magnético},
\begin{equation}
R_m \equiv \frac{U L}{\eta},
\end{equation}
donde $U$ y $L$ son escalas características de velocidad y longitud del sistema,
respectivamente, y $\eta$ es la difusividad magnética.

Cuando $R_m \gg 1$, el término advectivo de la ecuación de inducción domina sobre la
difusión resistiva, justificando la validez del régimen de flujo congelado. En el
límite ideal, $R_m \rightarrow \infty$, la reconexión magnética queda suprimida a
escalas macroscópicas.

En los discos de acreción astrofísicos considerados en este trabajo se alcanzan
valores extremadamente grandes de $R_m$, por lo que la aproximación de MHD ideal es
adecuada a escala global.


% ---------------------------------------------------------
\subsection{El tensor de tensiones magnéticas y la fuerza \texorpdfstring{$\vect{J} \times \vect{B}$}{JxB}}
% ---------------------------------------------------------

En Magnetohidrodinámica, la interacción del campo magnético con la corriente eléctrica produce fuerzas que actúan sobre el plasma. Estas fuerzas las incluimos en la ecuación de momento:
\begin{equation}
\rho \frac{\mathrm{d}\vect{u}}{\mathrm{d}t} = -\nabla p + \vect{J} \times \vect{B},
\end{equation}
donde $\vect{J} \times \vect{B}$ es la fuerza de Lorentz por unidad de volumen y no estamos teniendo en cuenta el término gravitatorio.

Para entender su estructura, es útil reescribirla mediante el **tensor de tensiones magnéticas**, también llamado **Maxwell stress tensor**. Recordando la ley de Ampère en el régimen no relativista:
\begin{equation}
\vect{J} = \frac{1}{\mu_0} \nabla \times \vect{B},
\end{equation}
y usando la identidad vectorial
\begin{equation}
(\nabla \times \vect{B}) \times \vect{B} = (\vect{B} \cdot \nabla) \vect{B} - \frac{1}{2} \nabla B^2,
\end{equation}
se puede escribir:
\begin{equation}
\vect{J} \times \vect{B} = \frac{1}{\mu_0} \left[ (\vect{B} \cdot \nabla) \vect{B} - \frac{1}{2} \nabla B^2 \right].
\label{eq:lorentz_tensor}
\end{equation}

Esta expresión muestra que la fuerza magnética tiene **dos contribuciones con interpretaciones físicas distintas**:

\begin{enumerate}
\item \textbf{Tensión magnética} (\textit{rubber band term}):
\begin{equation}
\vect{F}_\text{tension} = \frac{1}{\mu_0} (\vect{B} \cdot \nabla) \vect{B}.
\end{equation}
Actúa a lo largo de las líneas de campo, intentando mantenerlas rectas y resistiendo su estiramiento o curvatura. Se puede imaginar como un elástico que tiende a retraer la línea de campo si se deforma.

\item \textbf{Presión magnética} (\textit{inflating balloon term}):
\begin{equation}
\vect{F}_\text{pressure} = - \nabla \left( \frac{B^2}{2 \mu_0} \right).
\end{equation}
Actúa perpendicularmente a las líneas de campo, como la presión de un gas, empujando el plasma hacia afuera donde la densidad de campo es mayor. Es responsable de la expansión del plasma en regiones de alta energía magnética.
\end{enumerate}

Combinando ambos términos, se puede formalizar mediante el **tensor de tensiones magnéticas**:
\begin{equation}
\mathsf{T}_{ij} = \frac{1}{\mu_0} \left( - \frac{1}{2} B^2 \delta_{ij} + B_i B_j \right),
\end{equation}
de modo que la fuerza de Lorentz se obtiene como la divergencia del tensor:
\begin{equation}
\vect{J} \times \vect{B} = \nabla \cdot \mathsf{T}.
\end{equation}

\medskip
\noindent
\textbf{Plasma beta.}

La importancia relativa del campo magnético frente a la presión térmica del plasma
se caracteriza mediante el parámetro adimensional \textit{plasma beta},
\begin{equation}
\beta \equiv \frac{p}{\displaystyle \frac{B^2}{2\mu_0}}.
\end{equation}

Cuando $\beta \gg 1$, la presión térmica domina energéticamente la dinámica del
plasma, aunque el campo magnético puede seguir controlando la transferencia de
momento angular mediante tensiones magnéticas. En el régimen opuesto, $\beta \ll 1$,
la dinámica está directamente gobernada por el campo magnético.

En los discos de acreción analizados en este trabajo se cumple típicamente
$\beta > 1$, de modo que el campo magnético es energéticamente subdominante pero
dinámicamente esencial, condición clave para el desarrollo de la inestabilidad
magnetorrotacional.
\section{Inestabilidad magnetorrotacional como Motor de la Acreción}

% =========================================================
\subsection{Introducción fenomenológica: el ``muelle magnético''}
% =========================================================

La inestabilidad magnetorrotacional (MRI) puede entenderse de forma
intuitiva mediante una analogía introducida por Balbus y Hawley \cite{Balbus1991a}, que
ilustra el papel fundamental de las tensiones magnéticas en discos de
acreción en rotación diferencial.

Consideremos dos parcelas de gas situadas en órbitas casi circulares a
radios ligeramente distintos dentro del disco. En ausencia de campo
magnético, ambas conservarían su momento angular específico y
permanecerían en órbitas estables. Sin embargo, en el régimen de MHD
ideal, las líneas de campo magnético están congeladas en el fluido, de
modo que las parcelas quedan unidas por una línea de
campo, análoga a un muelle elástico. La fuerza restauradora asociada a
este muelle es la tensión magnética, descrita por el término
$(\vect{B}\cdot\nabla)\vect{B}$.
\\\\
En un disco kepleriano, donde la velocidad angular es $\Omega(R)\propto R^{-3/2}$, la parcela
interna orbita más rápido que la externa. Esta desunión cinemática
estira la línea de campo, generando una tensión que ejerce un torque
negativo sobre la parcela interna y positivo sobre la externa. Como
consecuencia, la parcela interna pierde momento angular y migra hacia
radios menores, mientras que la externa lo gana y se desplaza hacia
radios mayores. Este desplazamiento aumenta aún más el estiramiento del
campo, intensificando la tensión magnética y realimentando el proceso.
\begin{figure}[H] 
    \centering
    \includegraphics[width=0.3\textwidth]{Fotos/MRI.png} % Reducido a 0.45 del ancho del texto
    \caption{Representación esquemática de la inestabilidad magnetorrotacional (MRI). Las líneas de campo magnético verticales actúan como muelles elásticos que conectan parcelas de fluido a diferentes radios. Debido a la rotación diferencial ($\Omega$ decreciente con el radio), la línea de campo se estira, transfiriendo momento angular y provocando que la parcela interna caiga y la externa se aleje. Adaptado de Murphy (2014) \cite{Murphy2014}.}
    \label{fig:mri_muelle}
\end{figure}

Este mecanismo de realimentación positiva constituye la esencia física
de la MRI. La inestabilidad extrae energía de la rotación diferencial
del disco y no requiere campos magnéticos intensos: basta un campo débil
que acople regiones separadas radialmente sin rigidizar
el fluido, esto último ocurre cuando el campo es demasiado intenso \cite{Ji2013}.

% =========================================================
\subsection{Estado de equilibrio del disco magnetizado}
% =========================================================

El análisis de la inestabilidad magnetorrotacional se realiza en torno a
un estado de equilibrio idealizado. Consideramos un fluido conductor descrito
por la MHD ideal, en rotación diferencial alrededor de un objeto central
de masa $M$, utilizando coordenadas cilíndricas $(R,\phi,z)$.

En el equilibrio no perturbado, el flujo es puramente azimutal,
\begin{equation}
\vect{u}_0 = R\,\Omega(R)\,\hat{\boldsymbol{\phi}},
\end{equation}
y la rotación está dominada por el potencial gravitatorio central,
conduciendo a un perfil kepleriano,
\begin{equation}
\Omega(R) = \sqrt{\frac{GM}{R^3}},
\end{equation}
que satisface $\mathrm{d}\Omega/\mathrm{d}R < 0$, condición fundamental
para el desarrollo de la MRI.

El campo magnético de equilibrio se asume débil y con una componente
vertical uniforme,
\begin{equation}
\vect{B}_0 = B_0\,\hat{\boldsymbol{z}},
\end{equation}
tal que el plasma se encuentra en el régimen $\beta \gg 1$. Consideramos el caso en el que por tanto el campo no altera el equilibrio hidrodinámico global, pero es
dinámicamente relevante a través de las tensiones magnéticas.

El estudio se simplifica adoptando un marco local co--rotante centrado
en un radio $R_0$, conocido como \textit{shearing sheet} \cite{Hawley1991}. En este marco,
las coordenadas son
\begin{equation}
x = R - R_0, \qquad y = R_0(\phi - \Omega_0 t),
\end{equation}
Estas son válidas bajo la hipótesis $|x| \ll R_0$. La coordenada $x$ describe
desplazamientos radiales pequeños, mientras que $y$ representa
desplazamientos azimutales medidos en un sistema corrotante con el
disco a la velocidad angular local $\Omega_0$.
Estas nos permiten aproximar la rotación diferencial como un flujo de cizalla
lineal,
\begin{equation}
\vect{u}_0 = - q\,\Omega_0\,x\,\hat{\boldsymbol{y}},
\end{equation}
donde
\begin{equation}
q \equiv - \left.\frac{\mathrm{d}\ln\Omega}{\mathrm{d}\ln R}\right|_{R_0},
\end{equation}
con $q=3/2$ para un disco kepleriano.

% =========================================================
\subsection{Análisis de estabilidad lineal}
% =========================================================

En esta sección se formaliza la intuición física introducida
anteriormente mediante un análisis de estabilidad lineal de las
ecuaciones de la magnetohidrodinámica ideal. El objetivo es determinar
si pequeñas perturbaciones introducidas sobre el estado de equilibrio
definido en la Sección~3.2 crecen o decaen con el tiempo.

% ---------------------------------------------------------

% ---------------------------------------------------------

Partimos de las ecuaciones ya desarrolladas anteriormente de la MHD ideal para un fluido
incompresible:
\begin{align}
\nabla \cdot \vect{u} &= 0, \label{eq:incompresibilidad} \\
\frac{\partial \vect{u}}{\partial t}
+
(\vect{u}\cdot\nabla)\vect{u}
&=
-\frac{1}{\rho}\nabla p
-
\nabla \Phi
+
\frac{1}{\mu_0\rho}
\left[
(\vect{B}\cdot\nabla)\vect{B}
-
\nabla\left(\frac{B^2}{2}\right)
\right],
\label{eq:momento} \\[4pt]
\frac{\partial \vect{B}}{\partial t}
&=
\nabla \times (\vect{u} \times \vect{B}),
\label{eq:induccion} \\[4pt]
\nabla \cdot \vect{B} &= 0.
\label{eq:divB}
\end{align}
Estas ecuaciones las evaluaremos en el marco local corrotante definido en la
Sección~3.2, donde la rotación diferencial del disco aparece como un
flujo de cizalla lineal.

% ---------------------------------------------------------

% ---------------------------------------------------------

Introducimos pequeñas perturbaciones sobre el estado de equilibrio,
\begin{align}
\vect{u} &= \vect{u}_0 + \delta\vect{u}, \\
\vect{B} &= \vect{B}_0 + \delta\vect{B}, \\
p &= p_0 + \delta p,
\end{align}
y suponemos que las perturbaciones son suficientemente pequeñas como
para despreciar términos de segundo orden.

En el marco del \textit{shearing sheet}, y restringiéndonos a
perturbaciones axisimétricas, se buscan soluciones en forma de modos
normales,
\begin{equation}
\delta f(\vect{x},t) \propto e^{i(k_z z - \omega t)}.
\end{equation}
donde $k_z$ es el número de onda vertical y $\omega$ la frecuencia
compleja del modo.

% ---------------------------------------------------------

% ---------------------------------------------------------

Linealizando las ecuaciones de movimiento e inducción en el sistema rotante (siguiendo a Balbus \& Hawley \cite{Balbus1991a}), se obtiene el siguiente sistema para las componentes radial y azimutal de las perturbaciones:
\begin{align}
-i\omega\,\delta u_x - 2\Omega_0\,\delta u_y
&= -\frac{ik_z B_0}{\mu_0\rho}\,\delta B_x, \label{eq:lin_ux} \\
-i\omega\,\delta u_y + (2-q)\Omega_0\,\delta u_x
&= -\frac{ik_z B_0}{\mu_0\rho}\,\delta B_y, \label{eq:lin_uy} \\
-i\omega\,\delta B_x &= ik_z B_0\,\delta u_x, \label{eq:lin_Bx} \\
-i\omega\,\delta B_y &= ik_z B_0\,\delta u_y
- q\Omega_0\,\delta B_x. \label{eq:lin_By}
\end{align}

% ---------------------------------------------------------

% ---------------------------------------------------------

A partir de las ecuaciones de inducción
(\ref{eq:lin_Bx}) y (\ref{eq:lin_By}) se obtiene directamente:
\begin{align}
\delta B_x &= \frac{k_z B_0}{\omega}\,\delta u_x, \label{eq:Bx_elim} \\
\delta B_y &= \frac{k_z B_0}{\omega}\,\delta u_y
- \frac{q\Omega_0}{\omega}\,\delta B_x. \label{eq:By_elim}
\end{align}

Sustituyendo (\ref{eq:Bx_elim}) en (\ref{eq:By_elim}), resulta:
\begin{equation}
\delta B_y =
\frac{k_z B_0}{\omega}\,\delta u_y
- \frac{q\Omega_0 k_z B_0}{\omega^2}\,\delta u_x.
\label{eq:By_final}
\end{equation}

Introduciendo ahora estas expresiones en las ecuaciones de movimiento
(\ref{eq:lin_ux}) y (\ref{eq:lin_uy}), y definiendo la velocidad de
Alfvén
\begin{equation}
v_A \equiv \frac{B_0}{\sqrt{\mu_0\rho}},
\end{equation}

Sustituyendo \eqref{eq:Bx_elim} en \eqref{eq:lin_ux}:

\begin{equation}
-i\omega\,\delta u_x - 2\Omega_0\,\delta u_y
= - i \frac{k_z^2 v_A^2}{\omega}\,\delta u_x .
\end{equation}

Multiplicando por $i\omega$:

\begin{equation}
(\omega^2 - k_z^2 v_A^2)\,\delta u_x
- 2 i \Omega_0 \omega\,\delta u_y
= 0 .
\label{eq:sys1}
\end{equation}

Ahora sustituimos \eqref{eq:By_final} en \eqref{eq:lin_uy}:

\begin{align}
-i\omega\,\delta u_y + (2-q)\Omega_0\,\delta u_x
&=
- i \frac{k_z^2 v_A^2}{\omega}\,\delta u_y
+ \frac{q\Omega_0 k_z^2 v_A^2}{\omega^2}\,\delta u_x .
\end{align}

Multiplicando por $\omega^2$ y reordenando términos:

\begin{equation}
\left[(2-q)\Omega_0\omega^2
- q\Omega_0 k_z^2 v_A^2\right]\delta u_x
+ \omega\left(k_z^2 v_A^2 - \omega^2\right)\delta u_y
= 0 .
\label{eq:sys2}
\end{equation}

Las ecuaciones \eqref{eq:sys1} y \eqref{eq:sys2} constituyen un sistema homogéneo
para $(\delta u_x,\delta u_y)$. La condición de existencia de soluciones no
triviales es que el determinante se anule:

\begin{equation}
\begin{vmatrix}
\omega^2 - k_z^2 v_A^2
&
-2 i \Omega_0 \omega
\\[6pt]
(2-q)\Omega_0\omega^2 - q\Omega_0 k_z^2 v_A^2
&
\omega(k_z^2 v_A^2 - \omega^2)
\end{vmatrix}
= 0 .
\end{equation}

Tras una simplificación algebraica directa se obtiene la relación de dispersión (conforme a los resultados de Balbus \cite{Balbus2003} y Kulsrud \cite{Kulsrud2005}):

\begin{equation}
\omega^4
- \omega^2\left(\kappa^2 + 2k_z^2 v_A^2\right)
+ k_z^2 v_A^2
\left(
k_z^2 v_A^2 + \frac{d\Omega^2}{d\ln R}
\right)
= 0 ,
\end{equation}

donde hemos definido la frecuencia epicíclica

\begin{equation}
\kappa^2 = 2(2-q)\Omega_0^2,
\end{equation}

y utilizado

\begin{equation}
\frac{d\Omega^2}{d\ln R} = -2q\Omega_0^2 .
\end{equation}

Esta es la relación de dispersión general de la inestabilidad
magnetorrotacional para perturbaciones axisimétricas con propagación vertical.

% Asegúrate de tener \usepackage{float} en el preámbulo
\begin{figure}[H]
    \centering
    \includegraphics[width=0.55\textwidth]{Fotos/tasacrecimiento.png} 
    \caption{Tasa de crecimiento normalizada $\gamma/\Omega$ de la MRI en función del número de onda vertical $k_z v_A/\Omega$ para un perfil kepleriano ($q=1.5$). La curva muestra el rango de inestabilidad donde $\omega^2 < 0$, alcanzando un máximo de $\gamma \approx 0.75\Omega$ y desapareciendo en el punto de corte por tensión magnética. Gráfica generada numéricamente de la relación de dispersión de Balbus \& Hawley (1991) \cite{Balbus1991a}.}
    \label{fig:mri_grafica_propia}
\end{figure}

% ---------------------------------------------------------

% ---------------------------------------------------------

La estabilidad del sistema está determinada por el signo de $\omega^2$.
Si alguna de las soluciones de la ecuación de dispersión anterior satisface $\omega^2 < 0$, la frecuencia es puramente imaginaria y la perturbación crece exponencialmente con el tiempo.

Analizando el término independiente de la relación de dispersión, se observa que el factor crítico es:
\begin{equation}
k_z^2 v_A^2 + \frac{\mathrm{d}\Omega^2}{\mathrm{d}\ln R}.
\end{equation}

Dado que $k_z^2 v_A^2 > 0$, la aparición de una raíz inestable ($\omega^2 < 0$) requiere necesariamente que:
\begin{equation}
\frac{\mathrm{d}\Omega^2}{\mathrm{d}R} < 0.
\label{eq:condicion_mri}
\end{equation}

Esta condición, derivada originalmente por Balbus \& Hawley \cite{Balbus1991a}, es notablemente simple: cualquier flujo magnetizado en rotación diferencial cuya velocidad angular decrezca con el radio es inestable frente a la MRI, independientemente de la intensidad del campo, siempre que éste sea dinámicamente débil. 

En particular, para un disco kepleriano, donde $\Omega \propto R^{-3/2}$, se cumple inequívocamente la condición \eqref{eq:condicion_mri}. Es importante notar que esta condición es mucho más restrictiva que el criterio de Rayleigh para estabilidad hidrodinámica ($\mathrm{d}(R^2\Omega)^2/\mathrm{d}R > 0$), lo que permite la existencia de turbulencia en sistemas que, de otro modo, serían estables.



% ---------------------------------------------------------
\subsubsection{El límite de campo débil y la paradoja de la MRI}
% ---------------------------------------------------------

Un rasgo notable de la inestabilidad magnetorrotacional es que no requiere campos magnéticos intensos para operar. De hecho, la condición de inestabilidad $\mathrm{d}\Omega/\mathrm{d}R < 0$ se cumple en prácticamente todos los discos de acreción astrofísicos.

Esto conduce a una aparente paradoja: ¿cómo puede un campo magnéticamente débil desencadenar una inestabilidad dinámica tan eficiente? La clave reside en el papel de la tensión magnética. El campo actúa como un elemento elástico que conecta regiones del fluido en rotación diferencial, permitiendo la transferencia de momento angular.

Si el campo es excesivamente intenso, la tensión magnética rigidiza el fluido (haciendo que la velocidad de Alfvén $v_A$ sea muy alta) y suprime los desplazamientos relativos; matemáticamente, esto ocurre cuando $k_z^2 v_A^2$ supera el valor absoluto del gradiente de velocidad angular. Por el contrario, si el campo es demasiado débil, el acoplamiento magnético resulta ineficiente. 

La MRI opera en el régimen intermedio, donde el campo es dinámicamente relevante pero permite la deformación de las líneas de fuerza. Este régimen es el que se espera de forma natural en discos de acreción reales \cite{Kulsrud2005}.

% ---------------------------------------------------------
\subsection{Saturación no lineal y turbulencia MHD}
% ---------------------------------------------------------

El análisis presentado hasta ahora se basa en un tratamiento lineal. Sin embargo, la MRI conduce a un crecimiento exponencial rápido, rompiendo el régimen lineal en escalas temporales del orden del tiempo orbital, $\tau \sim \Omega_0^{-1}$.

Una vez que las perturbaciones alcanzan amplitudes finitas, el sistema entra en un régimen no lineal de turbulencia magnetohidrodinámica (MHD). En este estado, el campo magnético es continuamente amplificado y distorsionado, manteniendo un transporte de momento angular estadísticamente estacionario \cite{Balbus2003}.



La consecuencia física más importante de esta turbulencia es la aparición de esfuerzos de Maxwell:
\begin{equation}
T_{R\phi} \sim -\frac{B_R B_\phi}{\mu_0},
\end{equation}
que actúan como el mecanismo efectivo de transporte de momento angular que se buscaba en modelos fenomenológicos como el de Shakura \& Sunyaev. Estos esfuerzos permiten que la materia pierda momento angular y caiga hacia el objeto central, resolviendo de manera natural el problema de la viscosidad. 

La MRI proporciona así un mecanismo autosostenido y robusto que conecta la dinámica microscópica del campo magnético con la evolución macroscópica del sistema de acreción \cite{Ji2013}.
\section{De la acreción a la eyección}
Hasta ahora hemos visto que la MRI es capaz de explicar cómo el gas cae al objeto central que orbita, a pesar de que el campo magnético es, inicialmente, débil, local y desordenado. Sin embargo, en algunos de estos objetos astrofísicos se forman los llamados jets de acreción. Estos jets expulsan grandes cantidades de materia interestelar al espacio. No obstante, para que estos jets se formen, es necesario tener un campo magnético coherente a gran escala. Entonces, ¿cómo puede un disco de acreción dominado por la turbulencia MHD generar un campo magnético coherente a gran escala? Pues bien, en la siguiente subección trataremos de resolver esta problemática.

% =========================================================
\subsection{La turbulencia MRI como un dinamo rápido}
% =========================================================

De manera general, el mecanismo de funcionamiento de un dinamo nos permite transformar energía mecánica en magnética. Una manera sencilla de entender cómo la turbulencia actúa como un dinamo rápido (un dinamo rápido es un dinamo que gira a altas revoluciones) es mediante el llamado mencanismo Stretch-Twist-Fold (STF) introducido por el físico bieloruso Yákov Zeldóvich. 
\begin{figure}[h]
    \centering
    \includegraphics[width=0.5\linewidth]{Fotos/STF.png}
    \caption{Mecanismo Stretch-Twist-Fold}
\end{figure}
Este modelo refleja de una manera muy clara las claves necesarias para que un dinamos sea eficiente. Estas claves son:
\begin{description}
    \item[1.] Estiramiento del campo magnético por el cizallamiento del flujo.
    \item[2.] La necesidad de movimientos tridimensionales que eviten cancelaciones por el retorcimiento del flujo.
    \item[3.] Porcesos difusivos para hacer el ciclo irreversible.
\end{description}
Teniendo esto en cuenta, la amplificación del campo magnético que es inicialmente débil, se produce en escalas temporales comparables al tiempo dinámico del flujo, el tiempo en el que las propiedades del fluido cambian.

Todas estas características están presentes en la turbulencia generada por la MRI, por lo que se le puede considerar un dinamo rápido a la turbulencia MRI capaz de amplificar los campos magnéticos de manera eficiente.

% =========================================================
\subsection{Descripción del campo medio y generación de campos a gran escala}
% =========================================================

La teoría de dinamos de campo medio pemite entender cómo los campos magnéticos grandes y coherentes pueden generarse a partir de la interacción de campos turbulentos más pequeños.

En una primera instancia, Eugene Parker propuso la idea de que un campo magnético poloidal que surja ia partir de las fuerzas de Coriolis, las cuales pueden convertir un campo toroidal en poloidal, se podía describir introduciendo un término en la ecuación de inducción.
\begin{equation}
    \frac{\partial\langle\vect{B}_{p}\rangle}{\partial t}=\vect{\nabla}\times(\alpha\langle\vect{B}_{tor}\rangle+\dots)
\end{equation}
Esto solo es aplicable a los campos medios porque la fuerza electromagnética inducida ($\vect{u}\times\vect{B}$) nunca tendría una componente paralela al campo magneético. Recordemos que en la teoría de campos medios, las variables se descomponen en la suma del valor y de las fluctuaciones, al igual que en los flujos turbulentos. Así:
\begin{align*} 
    &\vect{u}=\langle\vect{u}\rangle+\vect{u}' &\vect{B}=\langle\vect{B}\rangle+\vect{B}'
\end{align*}
Entonces, tomando el promedio de la ecuación de la inducción tenemos
\begin{equation}
     \frac{\partial\langle\vect{B}\rangle}{\partial t}=\vect{\nabla}\times(\langle\vect{u}\rangle\times\langle\vect{B}\rangle+\langle\vect{\epsilon}\rangle-\eta\langle\vect{J}\rangle)
\end{equation}
Donde $\langle\vect{\epsilon}\rangle=\langle\vect{u}'\times\vect{B}'\rangle$ es la fuerza electromotriz media y $\eta\langle\vect{J}\rangle$ es el término resistivo que surge simplemente de que $\nabla\times B=J$. Ahora, nuestro objetivo es encontrar una forma para $\langle\vect{\epsilon}\rangle$. Para ello, vamos a asumir que $\langle\vect{\epsilon}\rangle$ se puede expresar en potencias de los gradientes del campo magnético medio.
\begin{equation*}
    \epsilon_{i}=\alpha_{ij}(\hat{\vect{g}},\hat{\vect{\Omega}},\langle\vect{B}\rangle,\dots)\langle B_{j}\rangle+\beta_{ijk}(\hat{\vect{g}},\hat{\vect{\Omega}},\langle\vect{B}\rangle,\dots)\frac{\partial \langle B_{j}\rangle}{\partial x_{j}}
\end{equation*}
Donde $\alpha_{ij}$ y $\beta_{ijk}$ son los coeficientes de transporte turbulento, los cuales dependen de la estratificación, velocidad angular y módulo del campo magnético medio. Los puntos de la ecuación indican que los coeficientes también podrían depender de relaciones que involucren la escala pequeña del campo magnético. Es posible calcular estos coeficientes y seguir desarrollando la expresión, pero por simplicidad vamos a proporcionar la expresión final para $\langle\vect{\epsilon}\rangle$.
\begin{equation}
    \langle\vect{\epsilon}\rangle=\alpha\langle\vect{B}\rangle-\beta_{t}\langle\vect{J}\rangle
\end{equation}
En contextos astrofísicos, generalmente interesan las soluciones en geometrías esféricas o en discos.

Las dinamos de campo medio se dividen tradicionalmente en dos grupos. Dinamos $\alpha\Omega$ y dinamos $\alpha^{2}$. El efecto se refiere a la amplificación del campo toroidal por cizallamiento (es decir, rotación diferencial). Dicho cizallamiento ocurre de forma natural en las galaxias de disco, ya que son sistemas de rotación diferencial. Sin embargo, también es necesario regenerar el campo poloidal. Los campos magnéticos a gran escala pueden generarse únicamente por el efecto $\alpha^{2}$, por lo que también el campo toroidal debe generarse por este efecto.
\newpage
\begin{figure}[h]
    \centering
    \includegraphics[width=0.75\linewidth]{Fotos/c.png}
    \caption{Regeneración mutua de campos poloidales y toroidales en el caso del $\alpha\Omega$ dinamo (izquierda) y del $\alpha^{2}$ dinamo (derecha).}
\end{figure}
Para ilustralos vamos a considerar la forma más simple de las ecuaciones de dinamo de campo medio axisimétricas apropiadas para un disco delgado.
\begin{align*}
    &\langle\dot{\vect{B}}_{R}\rangle=-(\alpha\langle\vect{B}_{\phi}\rangle)'+\beta_{T} \langle\vect{B}_{R}\rangle'' &\langle\dot{\vect{B}}_{\phi}\rangle=S\langle\vect{B}_{R}\rangle+\beta_{T} \langle\vect{B}_{\phi}\rangle''
\end{align*}
Donde los $'$ representan derivadas respecto a $z$, el punto representa derivadas temporales, $\alpha=\alpha_{0}f_{\alpha}(z)$ es un perfil axisimétrico para $\alpha$ y $S=Rd\Omega/dR$ es la cizalladura radial en el disco. En este contexto se pueden definir dos parámetros adimensionales.
\begin{align*}
    &C_{\Omega}=\frac{Sh^{2}}{\beta_{T}} &C_{\alpha}=\frac{\alpha_{0}h}{\beta_{T}}
\end{align*}
Donde $h$ es una medida de la altura del disco. Estos parámetros miden la intensidad de los efectos de la cizalladura y de $\alpha$. 

En general $|C_{\Omega}|\gg C_{\alpha}$. La generación de soluciones axisimétricas por el dinamo está controlada por un parámetro llamado número dinamo
\begin{equation}
    D=C_{\Omega}C_{\alpha}
\end{equation}
El crecimiento exponencial del campo magnético viene mediado por este parámetro. Cuando $|D|>D_{crit}$ sucede este crecimiento exponencial.

A partir del crecimiento exponencial del campo toroidal, por el efecto $\alpha^{2}$ este campo hace crecer también el campo poloidal. Es precisamente este campo el que provoca la aparición de los jets en los discos de acreción.
\newpage
\begin{figure}[h]
    \centering
    \includegraphics[width=0.35\linewidth]{Fotos/a.png}
    \caption{Jets en la dirección poloidal en el disco de acreción de un agujero negro}
\end{figure}


% =========================================================
\subsection{Simulaciones y jets}
% =========================================================

Cuando la teoría de dinamos se aplicó a los discos de acreción, inicialmente se consideraba un ejemplo más de los numerosos cuerpos astrofísicos que albergan dinamos. Sin embargo, con el descubrimiento de la inestabilidad magneto-rotacional se hizo evidente que los campos magnéticos son esenciales para sostener la turbulencia en los discos de acreción. Esto da lugar a un sistema autoexcitado en el que el campo magnético necesario para la MRI puede regenerarse continuamente mediante la acción del dinamo.

Simulaciones numéricas han mostrado que este sistema puede funcionar como un dinamo a pequeña escala si no existe estratificación vertical y como un dinamo de gran escala cuando hay estratificación. Estas simulaciones también permiten identificar efectos como el efecto $\alpha$ que, como hemos visto, interviene en la regeneración de campos magnéticos a gran escala.

Desde el punto de vista astrofísico, la presencia de campos magnéticos de gran escala es fundamental para los outflows observados en todos los discos de acreción. La aceleración magnetocentrífuga es uno de los mecanismos que puede impulsar estos flujos y los campos magnéticos responsables pueden provenir tanto del material arrastrado durante la formación del disco como de la propia acción del dinamo. Simulaciones tridimensionales de discos de acreción muestran que la MRI puede mantener un campo magnético estable y generar flujos de salida que se asocian directamente con la presencia de campos de gran escala en el disco. Particularmente, la componente vertical de estos campos contribuye significativamente al aumento del transporte angular efectivo, reflejado en valores típicos del parámetro de viscosidad $\alpha\approx0,1$ según el modelo de Shakura-Sunyaev.
\newpage
\begin{figure}[h]
    \centering
    \includegraphics[width=0.58\linewidth]{Fotos/b.png}
    \caption{Simulación de un outflow de un disco de acreción dinamoactivo impulsado por una combinación de presión y aceleración magnetocentrífuga}
\end{figure}

\newpage
% =========================================================




\bibliographystyle{plain} % O usa 'apalike' si prefieres (Autor, Año)
\bibliography{referencias}
\end{document}